\documentclass{article}
\usepackage[siunitx, americanresistors, americanvoltages, americancurrents, cuteinductors]{circuitikz}
\usepackage{circuitikz}
\pagestyle{empty}
\usepackage{xspace}


%___Circuit#1
\begin{document}
\begin{circuitikz}
\node[ground] at (0,0) (gnd1) {};
\draw (0,0)
to[V<=1<\volt>, invert] (0,5)
to[resistor, R=100<\kilo\ohm>] (5,5)
to[resistor, R=10<\kilo\ohm>] (10,5) -- (10,5)
to[resistor, R=50<\kilo\ohm>] (10,0) -- (10,0)
to[resistor, R=1<\kilo\ohm>] (5,0) -- (5,0)
to[resistor, R=25<\kilo\ohm>] (0,0)
-- (0,0);
\end{circuitikz}
\clearpage



%___Circuit#2
\begin{circuitikz}
\node[ground] at (5,0) (gnd1) {};
\draw (5,5) to[ospst=(t at 0s)] (10,5);
\draw (0,0) to[V<=1<\volt>, invert] (0,5) -- (5,5)
to[C=9<\nano\farad>] (5,0) -- (0,0);
\draw (5,5) -- (10,5) to[resistor, R=1<\kilo\ohm>] 
(10,0) -- (5,0)
\end{circuitikz}
\clearpage



%___Circuit#3
\begin{circuitikz}
\node[ground] at (0,0) (gnd1) {};
\draw (0,0)
to[vsourcesin,Hz=60<\hertz>,invert] (0,5)
to[resistor,R=100<\kilo\ohm>] (10,5)
to[inductor,L=2<\milli\henry>] (10,0) -- (10,0)
to[capacitor,C=50<\micro\farad>] (0,0) -- (0,0);
\end{circuitikz}
\end{document}